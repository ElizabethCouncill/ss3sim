\documentclass[12pt]{article}
\usepackage{geometry} 
\geometry{letterpaper}
\usepackage{graphicx}
\usepackage{Sweave}

% \VignetteIndexEntry{How to use the ss3sim package to run simulations in SS3}
% \VignetteKeyword{metapopulations}
% \VignetteKeyword{ecology}

%\usepackage[round]{natbib} 
%\bibliographystyle{apalike}   
%\bibpunct{(}{)}{;}{a}{}{;}   

\title{How to use the \texttt{ss3sim} package to run\\simulations in SS3}
\author{Sean C. Anderson and others to follow}
\date{}

\begin{document}
\maketitle

\noindent

\begin{Schunk}
\begin{Sinput}
> # install.packages("devtools")
> # devtools::install_github("ss3sim", username="seananderson")
> library(ss3sim)
\end{Sinput}
\end{Schunk}

\noindent
make folders

\noindent
make oms and ems in ...

\begin{Schunk}
\begin{Sinput}
> copy_models(model_dir = "oms", type = "om")
> copy_models(model_dir = "ems", type = "em")
\end{Sinput}
\end{Schunk}

\noindent
or if you were only responsible for 1:50:

\begin{Schunk}
\begin{Sinput}
> copy_models(model_dir = "oms", type = "om", iterations = 1:50)
\end{Sinput}
\end{Schunk}

\noindent
This creates the structure:

\begin{verbatim}
  sc1-cod/1/om
  sc1-cod/1/em
  sc1-cod/2/om
  sc1-cod/2/em
  ...
\end{verbatim}

\begin{Schunk}
\begin{Sinput}
> scenarios <- scan("mysenarios.txt", what = "character")
> run_scenario(scenarios, iterations = 1:50)
\end{Sinput}
\end{Schunk}

\noindent
Or, to test the operating model for the first scenario:

\begin{Schunk}
\begin{Sinput}
> run_scenario(scenarios[1], iterations = 1, type = "om")
\end{Sinput}
\end{Schunk}

%\bibliography{/Users/seananderson/Dropbox/tex/jshort,/Users/seananderson/Dropbox/tex/ref}
\end{document}


